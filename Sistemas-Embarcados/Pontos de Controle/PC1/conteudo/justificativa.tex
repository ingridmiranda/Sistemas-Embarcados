\section{Justificativa}

  De acordo com os dados do Censo de 2010, aproximadamente 45 milhões de pessoas declaram ter algum tipo de deficiência no Brasil. E um dos principais problemas enfrentados por essas pessoas consiste na falta de acessibilidade \cite{dadosIBGE}.
  Outra parcela da população que sofre com esse problema consiste nos idosos, pois o processo de envelhecimento causa, além da perda da força muscular, a redução da agilidade, coordenação, equilíbrio e mobilidade \cite{idosos}.
  Neste sentido, este projeto pode proporcionar soluções que facilitam o dia-a-dia dos seus usuários, em especial, os dos grupos citados anteriormente. Tornando, desse modo, as instalações residenciais mais acessíveis e facilitando o acesso à informação.
  Atualmente, há equipamentos com funcionalidades similares às deste projeto. Um deles consiste no Amazon Echo, que utilizando o serviço de voz Alexa, tem como aplicações tocar música, receber e enviar mensagens, fazer ligações e fornecer informações sobre esportes, clima e notícias instantaneamente \cite{echo}. Entretanto, o diferencial desta proposta consiste em adicionar como mais uma finalidade a automação residencial de alguns itens, tais como, acendimento de luzes e abertura de portas.

