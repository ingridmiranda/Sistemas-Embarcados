\section{Requisitos}

\subsection{Requisitos Funcionais}
\begin{enumerate}
  \item Todo a lógica do sistema será implementada usando a Raspberry Pi.
  \item O sistema deverá possuir uma interface externa de captura de áudio USB.
  \item O sistema deverá possuir um amplificador de potência externo ao RaspberryPi para permitir reprodução de áudio \cite{haro2014low}.
  \item O sistema terá conectividade com a web.
  \item O sistema estará configurado para interagir com entradas e saídas digitais.
  \item O sistema deverá possuir drivers externos de corrente para chavear dispositivos eletrônicos externos.
  \item O sistema irá possuir uma lógica de software para decodificar instruções de voz em strings.
  \item As entradas digitais deverão possuir uma lógica de circuito para detecar caso algum dispositivo seja desconectado.
  \item O sistema deverá possuir um sistema de detecção de falhas que envia as mesmas para o usuário por email..
  \item O sistema deve possuir uma tensão de entrada de 110V/220V afim de facilitar sua conexão com a rede de energia residencial.
\end{enumerate}

\subsection{Requisitos Não Funcionais}
\begin{enumerate}
  \item O sistema deve ser capaz de garantir uma mínima funcionalidade caso perca sua conexão com a internet.
  \item O sub-sistema de captura/decodificação de voz deve ser preciso o suficiente para que o usuário não precise ficar repetindo muitas vezes certo comando.
  \item O sistema deverá ter layout físico muito simples com um mínimo de botões para garantir que o usuário não precise interagir mecânicamente com o mesmo para operá-lo (salve durante processo de instalação e manutenção).
  \item O conjunto do amplificador de potência e o alto-falante deverão ter volume configurado por hardware (potenciômetro) e software afim de ajustar o SPL \textit{(Sound Pressure Level, Nível de Pressão Sonora) \cite{spl}} para melhor se adequar as necessidades do usuário..
\end{enumerate}
